% LANGUAGE / TYPESETTING
	%% Set the document language
	\usepackage[ngerman]{babel}
	
	%% Input of special characters e.g. áüó
	% Must be loaded before csquotes
	\usepackage[utf8]{inputenc}

	%% Glyph adjust,
	% e.g. allows the characters 'äúö' etc.
	% to be correctly copied from PDF.
	% If not loaded, 'auo' would be copied.
	% However, fontenc unsmoothes the default serif font
	%	(workaround: load a serif font package,
	%	e.g. lmodern or mathpazo)
	\usepackage[T1]{fontenc}


	%% Enables quotes to be typed as " " instead of `` ''	
	\usepackage[autostyle=true]{csquotes}	
	\MakeOuterQuote{"}
	
	%% Font selection
	% If serif font, choose one or more from:
%	\addtokomafont{disposition}{\rmfamily} % headings
%	\usepackage{lmodern} 		% (only text)
%	\usepackage[osf]{mathpazo} 	% (text and math)
%	\usepackage{eulervm}		% (only math)
	% If sans serif:
    \usepackage{helvet}
 	\renewcommand{\familydefault}{\sfdefault}	% (only text)

    % Scales default interline space to 1.5 of default value
    \renewcommand{\baselinestretch}{1.5}
	
	%% No widows (only one line at the start of a page)
	%  and no orphans (first line of paragraph at the end of a page)
	\widowpenalty10000						% No widows
	\clubpenalty10000						% No orphans
	
	
% MARGINS
	%% Set page margins
    \usepackage{showframe}
	\usepackage[top=2.5cm,bottom=2.5cm,
				left=2.5cm,right=2.5cm,
				bindingoffset=8mm,
				%showframe, % construction lines
				footskip=0.5cm, % footer/page.num. position
				]{geometry}	
										
	%% Removes paragraph indentation
	\setlength{\parindent}{0pt}

	
% HEADER / FOOTER
	% Clear header and footer before setting new ones!
	
	%% Clears header footer
	\usepackage{scrlayer-scrpage}
	\clearscrheadfoot
	
	%% Sets header									
%	\rohead[]{}		


% Space before footnote line
	\let\svfootnoterule\footnoterule
	\renewcommand\footnoterule{\vspace{1cm}\svfootnoterule}	
	
	%% Sets footer	
	% Right hand side, odd pages if double-sided print,
	% 	all pages for single-sided	
	\rofoot{\footnotesize{\thepage}}			
				
				
% MATH
	%% Required for some math functions and symbols
	\usepackage{amsmath}
%	\usepackage{amssymb}

	%% Equation 1.1, 1.2, 2.1, ... instead of
	%  eqn 1, 2, 3, ...
%	\numberwithin{equation}{section}

	%% SI units
%	\usepackage{siunitx}
%		\sisetup{
%			tight-spacing           = true,
%			round-mode              = places,
%	        round-precision         = 2,
%			table-align-text-pre	= false,
%			output-decimal-marker   = {,},
%			group-separator 	    = {.},
%			}

	%% Sets equation number of a line within
	%  a starred equation environment
%	\newcommand\numberthis{\addtocounter{equation}{1}\tag{\theequation}}


% GRAPHICS
	%% Enables icludegraphics
	\usepackage{graphicx}
	
	%% Subfigures, i.e. Figure 1a, 1b				
	\usepackage{subcaption}

	%% Sets caption format to bold
	\usepackage[labelfont=bf]{caption}	% Caption format

	%% Forces position of floats (e.g. tables, figures)
	\usepackage{float}					% Position of floats
	
	%% Sets path (folder) for graphics so that
	%  the folder doesn't have to be typed in
	%  in the \includegraphics command
	\graphicspath{	{./img/}
					}	
			
	%% Tikz package to generate graphics		
%	\usepackage{tikz}


% TABLES
	%% Prettier table compared to default
	\usepackage{tabu}
	\usepackage{booktabs}

	%% Enables rows spanning multiple lines
%	\usepackage{multirow}	
	
	%% For tables that go beyond one page
%%	\usepackage{longtable}

	%% Row style - allows the same style in a column
	%	to be applied across all rows
	% 	requires the array package
	\newcolumntype{+}{>{\global\let\currentrowstyle\relax}} 
	\newcolumntype{^}{>{\currentrowstyle}}
	\newcolumntype{*}{>{\bfseries}}
	\newcommand{\rowstyle}[1]{\gdef\currentrowstyle{#1}%
	  #1\ignorespaces}
	  
	  
% LISTS
	%% Adjusts indentation, etc in lists
	\usepackage{enumitem}
		% spacing b/w list items
		\setlist{nolistsep}	
		\setlist[enumerate,1]
				% list indentation
				{labelindent=0.5cm,leftmargin=*}
				
				
% TABLE OF STUFF
	%% Sets the style for the list of tables/figures
	\usepackage[titles]{tocloft}
		\newlength{\figlength}
		\renewcommand{\cftfigpresnum}{\figurename\enspace}
		\renewcommand{\cftfigaftersnum}{:\enspace}
		\settowidth{\figlength}{\cftfigpresnum\cftfigaftersnum}
		\addtolength{\cftfignumwidth}{\figlength}
		
		\newlength{\tablength}
		\renewcommand{\cfttabpresnum}{\tablename\enspace}
		\renewcommand{\cfttabaftersnum}{:}
		\settowidth{\tablength}{\cfttabpresnum\cfttabaftersnum}
		\addtolength{\cfttabnumwidth}{\tablength}
	
	
% CITATIONS / BIBLIOGRAPHY
	%% Citation package
	\usepackage[maxcitenames=10,
				backend=biber,
				sorting=nty,
				date=year,
				sortcites=true,
				]{biblatex}
				
	%% List of references
	\addbibresource{./preamble/ref.bib}
	
	%% Bibliography style -- here based on
	%  Zitierrichtlinie nach DIN 1505
	% Original:
% $Id: standard.bbx,v 1.6 2011/07/29 19:21:28 lehman stable $
% Angepasst:
% din.bbx, v2012-06-27, Michael Domhardt
% gekürzt nach Vorschlag von moewew vom 5. Nov. 2015, siehe: https://github.com/domhardt/BibLaTeX-DIN1505/issues/1#issuecomment-154122346


%\setlength{\bibinitsep}{\baselineskip}						% Dist. bw unlike initials
\setlength\bibhang{1em}

\DeclareNameAlias{default}{family-given}					% Last name, First name in bib
\renewcommand*{\mkbibnamefamily}[1]{\hsc{#1}}				% Small caps for last name
\renewcommand*{\multinamedelim}{\mbox{ }\addspace\addsemicolon\space}	% Semicolon b/w authors
\renewcommand*{\finalnamedelim}{\multinamedelim}			% Semicolon before last author
\renewcommand{\labelnamepunct}{\addcolon\space}				% Colon after last author
\renewcommand*{\finentrypunct}{}							% No dot at the end

\DeclareFieldFormat[inbook]{title}{#1\midsentence}			% No quote marks in chap title
\DeclareFieldFormat[inproceedings]{title}{#1\midsentence}	% No quote marks in the title
\DeclareFieldFormat{editortype}{\mkbibparens{#1}}			% Parentheses editor
\DeclareFieldFormat[thesis]{title}{\mkbibemph{#1}\midsentence}

\DefineBibliographyStrings{german}
							{urlseen = {Zugriff am}}
							
\defbibenvironment{bibliography}
	{\list{\printtext[labelnumberwidth]%
					{\printfield{labelnumber}}}
			{\setlength{\leftmargin}{\bibhang}%
			\setlength{\itemindent}{-\leftmargin}%
%			\setlength{\labelsep}{\biblabelsep}%
			\addtolength{\leftmargin}{1.75em}%
%			\setlength{\itemsep}{\bibitemsep}%
			\setlength{\parsep}{\bibparsep}
			}
	}
	{\endlist}
{\item}

% Removes comma before (Hrsg.)
\renewbibmacro*{editor+others}{%
  \ifboolexpr{
    test \ifuseeditor
    and
    not test {\ifnameundef{editor}}
  }
    {\printnames{editor}%
     \setunit{\space}% <- hier
     \usebibmacro{editor+othersstrg}%
     \clearname{editor}}
    {}}


% Defines new macro which issues \setunit to generate line breaks only (for URLs)
\newbibmacro*{bbx:parunit}{%
  \ifbibliography
    {\setunit{\bibpagerefpunct}\newblock
     \usebibmacro{pageref}%
     \clearlist{pageref}%
     \setunit{\adddot\par\nobreak}}
    {}}

% With previous code, sets URL on a new line
\renewbibmacro*{url+urldate}{%
  \usebibmacro{bbx:parunit}% Added
  \printfield{url}%
  \iffieldundef{urlyear}
    {}
    {\usebibmacro{bbx:parunit}%
     \printtext[]{\printurldate}}}

    
\DeclareBibliographyDriver{book}{% 
   \usebibmacro{bibindex}% 
   \usebibmacro{begentry}% 
   \usebibmacro{author/editor+others/translator+others}% 
		\setunit{\labelnamepunct}
   \newblock 
   \usebibmacro{maintitle+title}% 
		\newunit 
   		\printlist{language}% 
   		\newunit
   \newblock 
   \usebibmacro{byauthor}\newunit
   \newblock 
   \usebibmacro{editor+others}% 
   \newunit\newblock 
   \printfield{edition}% 
   \newunit 
   \iffieldundef{maintitle} 
     {\printfield{volume}% 
      \printfield{part}} 
     {}% 
   \newunit 
   \printfield{volumes}% 
   \newunit\newblock 
   \usebibmacro{series+number}% 
   \newunit\newblock 
   \printfield{note}% 
   \newunit\newblock 
   \usebibmacro{addendum+pubstate}%
   \setunit{\labelnamepunct}
   \newblock
   \usebibmacro{publisher+location+date}% 
   \newunit\newblock 
   \usebibmacro{chapter+pages}% 
   \newunit 
   \printfield{pagetotal}% 
   \newunit\newblock 
   \iftoggle{bbx:isbn} 
     {\printfield{isbn}} 
     {}% 
   \newunit\newblock 
   \usebibmacro{doi+eprint+url}% 
   \newunit\newblock 
   \usebibmacro{pageref}% 
   \usebibmacro{finentry}}

\DeclareBibliographyDriver{inbook}{%
  \usebibmacro{bibindex}%
  \usebibmacro{begentry}%
  \usebibmacro{author/translator+others}%
  \setunit{\labelnamepunct}\newblock
  \usebibmacro{title}%
  \newunit
  \printlist{language}%
  \newunit\newblock
  \usebibmacro{byauthor}%
  \newunit\newblock
  \usebibmacro{in:}%
  \usebibmacro{bybookauthor}%
  \newunit\newblock
  \usebibmacro{editor+others}% Herausgeber (Hrsg.) statt hrsg. von Herausgeber
  \setunit{\labelnamepunct}\newblock%
  \usebibmacro{maintitle+booktitle}%
  \newunit\newblock
  \printfield{edition}%
  \newunit
  \iffieldundef{maintitle}
    {\printfield{volume}%
     \printfield{part}}
    {}%
  \newunit
  \printfield{volumes}%
  \newunit\newblock
  \usebibmacro{series+number}%
  \newunit\newblock
  \printfield{note}%
  \newunit\newblock
  \usebibmacro{addendum+pubstate}%
  \setunit{\labelnamepunct}
  \newblock
  \usebibmacro{publisher+location+date}%
  \newunit\newblock
  \usebibmacro{chapter+pages}%
  \newunit\newblock
  \iftoggle{bbx:isbn}
    {\printfield{isbn}}
    {}%
  \newunit\newblock
  \usebibmacro{doi+eprint+url}%
  \newunit\newblock
  \usebibmacro{pageref}%
  \newunit\newblock
  \iftoggle{bbx:related}
    {\usebibmacro{related:init}%
     \usebibmacro{related}}
    {}%
  \usebibmacro{finentry}}

\DeclareBibliographyDriver{online}{% 
   \usebibmacro{bibindex}% 
   \usebibmacro{begentry}% 
   \usebibmacro{author/editor+others/translator+others}% 
   \setunit{\labelnamepunct}\newblock 
   \usebibmacro{title}% 
   \newunit\newblock 
   \usebibmacro{byauthor}% 
   \newunit\newblock 
   \usebibmacro{byeditor+others}% 
   \newunit 
   \printfield{note}% 
   \newunit\newblock 
   \printlist{organization}% 
   \newunit\newblock 
   \usebibmacro{date}% 
   \newunit\newblock 
   \iftoggle{bbx:eprint} 
     {\usebibmacro{eprint}} 
     {}% 
   \newunit\newblock 
   \usebibmacro{url+urldate}% 
   \setunit{\bibpagerefpunct}\newblock 
   \usebibmacro{pageref}% 
   \usebibmacro{finentry}}
   
\DeclareBibliographyDriver{inproceedings}{%
  \usebibmacro{bibindex}%
  \usebibmacro{begentry}%
  \usebibmacro{author/translator+others}%
  \setunit{\labelnamepunct}\newblock
  \textit{\usebibmacro{title}}%
  \newunit
  \printlist{language}%
  \newunit\newblock
  \usebibmacro{byauthor}%
  \newunit\newblock
  \usebibmacro{in:}%
  \usebibmacro{editor+others}%
  \setunit{\labelnamepunct}\newblock%
  \usebibmacro{maintitle+booktitle}%
  \usebibmacro{event+venue+date}%
%  \newunit\newblock
%  \newunit\newblock
%  \iffieldundef{maintitle}
%    {\printfield{volume}%
%     \printfield{part}}
%    {}%
%  \newunit
%  \printfield{volumes}%
%  \newunit\newblock
%  \usebibmacro{series+number}%
%  \newunit\newblock
%  \printfield{note}%
%  \newunit\newblock
%  \printlist{organization}%
%  \newunit
%  \usebibmacro{publisher+location+date}%
%  \newunit\newblock
%  \usebibmacro{chapter+pages}%
%  \newunit\newblock
%  \iftoggle{bbx:isbn}
%    {\printfield{isbn}}
%    {}%
%  \newunit\newblock
%  \usebibmacro{doi+eprint+url}%
%  \newunit\newblock
%  \usebibmacro{addendum+pubstate}%
%  \setunit{\bibpagerefpunct}\newblock
%  \usebibmacro{pageref}%
%  \newunit\newblock
%  \iftoggle{bbx:related}
%    {\usebibmacro{related:init}%
%     \usebibmacro{related}}
%    {}%
  \usebibmacro{finentry}}
   
% zusätzlicher Eintragstyp @standard
% geändert von @misc
\DeclareBibliographyDriver{standard}{%
   \usebibmacro{bibindex}% 
   \usebibmacro{begentry}% 
   \usebibmacro{author}% Nummer zuerst
   \setunit{\labelnamepunct}
   \newblock 
   \usebibmacro{title}%
   \newunit\newblock 
   \usebibmacro{addendum+pubstate}%
   \setunit{\labelnamepunct}
   \newblock
   \usebibmacro{publisher+location+date}%
   \setunit{\bibpagerefpunct}\newblock 
   \usebibmacro{pageref}% 
   \usebibmacro{finentry}}
	
	%% URL font: rmfamily for serif, , sffamily for sans serif
	\renewcommand{\UrlFont}{\rmfamily\footnotesize}
	
	
% DOCUMENT PROPERTIES
	%% Enables hyperlinks in PDF
	\usepackage{hyperref}
	
	\hypersetup{	pdftitle = {Erstellung einer studentischen Arbeit},
			pdfauthor = {Salehah},
			% Default zoom for PDF
			pdfstartview = {XYZ null null 1.0},
			% Make headings into bookmarks
			bookmarksnumbered = true,
			% Open bookmarks panel when PDF is opened
			bookmarksopen = true,
			% Make links coloured
			colorlinks,
			linkcolor={blue},
			citecolor={blue},
			urlcolor={red},
			}
