\subsubsection{Einstellungen in der Präambel}
Um die Literaturverzeichnis zu erstellen, muss man in der Präambel
das Paket \texttt{biblatex} laden.
Dazu sind die folgenden Optionen empfohlen.
Ansonsten kann man in die \texttt{biblatex} Dokumentation
anschauen.

\begin{verbatim}
\usepackage[backend=biber,
    sorting=nty,
    date=year,
    sortcites=true,
    ]{biblatex}
\end{verbatim}

\begin{center}
\begin{tabu} to \textwidth{>{\ttfamily}X[l] X[4l]}
	\toprule
	sorting=nty	& sorts entries by name, title, year\\
	date=year	& nur das Veröffentlichungsjahr wird gezeigt, und
		nicht das vollständige Datum\\
	sortcites=true	& obige \texttt{nty} Option wird umgesetzt\\
	\bottomrule
\end{tabu}
\end{center}

Die Informationen (Angaben zu Autor, Jahr, Titel, Auflage usw.)
werden in einer \texttt{.bib} Datei aufbewahrt und müssten
ebenfalls in der Präambel geladen werden.

\begin{verbatim}
\addbibresource{./preamble/ref.bib}
\end{verbatim}
	
Man kann an dieser Stelle auch den Stil der Bibliographie einstellen.
Z.B. nach APA, MLA, usw.
\begin{verbatim}
% Original:
% $Id: standard.bbx,v 1.6 2011/07/29 19:21:28 lehman stable $
% Angepasst:
% din.bbx, v2012-06-27, Michael Domhardt
% gekürzt nach Vorschlag von moewew vom 5. Nov. 2015, siehe: https://github.com/domhardt/BibLaTeX-DIN1505/issues/1#issuecomment-154122346


%\setlength{\bibinitsep}{\baselineskip}						% Dist. bw unlike initials
\setlength\bibhang{1em}

\DeclareNameAlias{default}{family-given}					% Last name, First name in bib
\renewcommand*{\mkbibnamefamily}[1]{\hsc{#1}}				% Small caps for last name
\renewcommand*{\multinamedelim}{\mbox{ }\addspace\addsemicolon\space}	% Semicolon b/w authors
\renewcommand*{\finalnamedelim}{\multinamedelim}			% Semicolon before last author
\renewcommand{\labelnamepunct}{\addcolon\space}				% Colon after last author
\renewcommand*{\finentrypunct}{}							% No dot at the end

\DeclareFieldFormat[inbook]{title}{#1\midsentence}			% No quote marks in chap title
\DeclareFieldFormat[inproceedings]{title}{#1\midsentence}	% No quote marks in the title
\DeclareFieldFormat{editortype}{\mkbibparens{#1}}			% Parentheses editor
\DeclareFieldFormat[thesis]{title}{\mkbibemph{#1}\midsentence}

\DefineBibliographyStrings{german}
							{urlseen = {Zugriff am}}
							
\defbibenvironment{bibliography}
	{\list{\printtext[labelnumberwidth]%
					{\printfield{labelnumber}}}
			{\setlength{\leftmargin}{\bibhang}%
			\setlength{\itemindent}{-\leftmargin}%
%			\setlength{\labelsep}{\biblabelsep}%
			\addtolength{\leftmargin}{1.75em}%
%			\setlength{\itemsep}{\bibitemsep}%
			\setlength{\parsep}{\bibparsep}
			}
	}
	{\endlist}
{\item}

% Removes comma before (Hrsg.)
\renewbibmacro*{editor+others}{%
  \ifboolexpr{
    test \ifuseeditor
    and
    not test {\ifnameundef{editor}}
  }
    {\printnames{editor}%
     \setunit{\space}% <- hier
     \usebibmacro{editor+othersstrg}%
     \clearname{editor}}
    {}}


% Defines new macro which issues \setunit to generate line breaks only (for URLs)
\newbibmacro*{bbx:parunit}{%
  \ifbibliography
    {\setunit{\bibpagerefpunct}\newblock
     \usebibmacro{pageref}%
     \clearlist{pageref}%
     \setunit{\adddot\par\nobreak}}
    {}}

% With previous code, sets URL on a new line
\renewbibmacro*{url+urldate}{%
  \usebibmacro{bbx:parunit}% Added
  \printfield{url}%
  \iffieldundef{urlyear}
    {}
    {\usebibmacro{bbx:parunit}%
     \printtext[]{\printurldate}}}

    
\DeclareBibliographyDriver{book}{% 
   \usebibmacro{bibindex}% 
   \usebibmacro{begentry}% 
   \usebibmacro{author/editor+others/translator+others}% 
		\setunit{\labelnamepunct}
   \newblock 
   \usebibmacro{maintitle+title}% 
		\newunit 
   		\printlist{language}% 
   		\newunit
   \newblock 
   \usebibmacro{byauthor}\newunit
   \newblock 
   \usebibmacro{editor+others}% 
   \newunit\newblock 
   \printfield{edition}% 
   \newunit 
   \iffieldundef{maintitle} 
     {\printfield{volume}% 
      \printfield{part}} 
     {}% 
   \newunit 
   \printfield{volumes}% 
   \newunit\newblock 
   \usebibmacro{series+number}% 
   \newunit\newblock 
   \printfield{note}% 
   \newunit\newblock 
   \usebibmacro{addendum+pubstate}%
   \setunit{\labelnamepunct}
   \newblock
   \usebibmacro{publisher+location+date}% 
   \newunit\newblock 
   \usebibmacro{chapter+pages}% 
   \newunit 
   \printfield{pagetotal}% 
   \newunit\newblock 
   \iftoggle{bbx:isbn} 
     {\printfield{isbn}} 
     {}% 
   \newunit\newblock 
   \usebibmacro{doi+eprint+url}% 
   \newunit\newblock 
   \usebibmacro{pageref}% 
   \usebibmacro{finentry}}

\DeclareBibliographyDriver{inbook}{%
  \usebibmacro{bibindex}%
  \usebibmacro{begentry}%
  \usebibmacro{author/translator+others}%
  \setunit{\labelnamepunct}\newblock
  \usebibmacro{title}%
  \newunit
  \printlist{language}%
  \newunit\newblock
  \usebibmacro{byauthor}%
  \newunit\newblock
  \usebibmacro{in:}%
  \usebibmacro{bybookauthor}%
  \newunit\newblock
  \usebibmacro{editor+others}% Herausgeber (Hrsg.) statt hrsg. von Herausgeber
  \setunit{\labelnamepunct}\newblock%
  \usebibmacro{maintitle+booktitle}%
  \newunit\newblock
  \printfield{edition}%
  \newunit
  \iffieldundef{maintitle}
    {\printfield{volume}%
     \printfield{part}}
    {}%
  \newunit
  \printfield{volumes}%
  \newunit\newblock
  \usebibmacro{series+number}%
  \newunit\newblock
  \printfield{note}%
  \newunit\newblock
  \usebibmacro{addendum+pubstate}%
  \setunit{\labelnamepunct}
  \newblock
  \usebibmacro{publisher+location+date}%
  \newunit\newblock
  \usebibmacro{chapter+pages}%
  \newunit\newblock
  \iftoggle{bbx:isbn}
    {\printfield{isbn}}
    {}%
  \newunit\newblock
  \usebibmacro{doi+eprint+url}%
  \newunit\newblock
  \usebibmacro{pageref}%
  \newunit\newblock
  \iftoggle{bbx:related}
    {\usebibmacro{related:init}%
     \usebibmacro{related}}
    {}%
  \usebibmacro{finentry}}

\DeclareBibliographyDriver{online}{% 
   \usebibmacro{bibindex}% 
   \usebibmacro{begentry}% 
   \usebibmacro{author/editor+others/translator+others}% 
   \setunit{\labelnamepunct}\newblock 
   \usebibmacro{title}% 
   \newunit\newblock 
   \usebibmacro{byauthor}% 
   \newunit\newblock 
   \usebibmacro{byeditor+others}% 
   \newunit 
   \printfield{note}% 
   \newunit\newblock 
   \printlist{organization}% 
   \newunit\newblock 
   \usebibmacro{date}% 
   \newunit\newblock 
   \iftoggle{bbx:eprint} 
     {\usebibmacro{eprint}} 
     {}% 
   \newunit\newblock 
   \usebibmacro{url+urldate}% 
   \setunit{\bibpagerefpunct}\newblock 
   \usebibmacro{pageref}% 
   \usebibmacro{finentry}}
   
\DeclareBibliographyDriver{inproceedings}{%
  \usebibmacro{bibindex}%
  \usebibmacro{begentry}%
  \usebibmacro{author/translator+others}%
  \setunit{\labelnamepunct}\newblock
  \textit{\usebibmacro{title}}%
  \newunit
  \printlist{language}%
  \newunit\newblock
  \usebibmacro{byauthor}%
  \newunit\newblock
  \usebibmacro{in:}%
  \usebibmacro{editor+others}%
  \setunit{\labelnamepunct}\newblock%
  \usebibmacro{maintitle+booktitle}%
  \usebibmacro{event+venue+date}%
%  \newunit\newblock
%  \newunit\newblock
%  \iffieldundef{maintitle}
%    {\printfield{volume}%
%     \printfield{part}}
%    {}%
%  \newunit
%  \printfield{volumes}%
%  \newunit\newblock
%  \usebibmacro{series+number}%
%  \newunit\newblock
%  \printfield{note}%
%  \newunit\newblock
%  \printlist{organization}%
%  \newunit
%  \usebibmacro{publisher+location+date}%
%  \newunit\newblock
%  \usebibmacro{chapter+pages}%
%  \newunit\newblock
%  \iftoggle{bbx:isbn}
%    {\printfield{isbn}}
%    {}%
%  \newunit\newblock
%  \usebibmacro{doi+eprint+url}%
%  \newunit\newblock
%  \usebibmacro{addendum+pubstate}%
%  \setunit{\bibpagerefpunct}\newblock
%  \usebibmacro{pageref}%
%  \newunit\newblock
%  \iftoggle{bbx:related}
%    {\usebibmacro{related:init}%
%     \usebibmacro{related}}
%    {}%
  \usebibmacro{finentry}}
   
% zusätzlicher Eintragstyp @standard
% geändert von @misc
\DeclareBibliographyDriver{standard}{%
   \usebibmacro{bibindex}% 
   \usebibmacro{begentry}% 
   \usebibmacro{author}% Nummer zuerst
   \setunit{\labelnamepunct}
   \newblock 
   \usebibmacro{title}%
   \newunit\newblock 
   \usebibmacro{addendum+pubstate}%
   \setunit{\labelnamepunct}
   \newblock
   \usebibmacro{publisher+location+date}%
   \setunit{\bibpagerefpunct}\newblock 
   \usebibmacro{pageref}% 
   \usebibmacro{finentry}}
\end{verbatim}


Ich habe bisher immer nach DIN 1505 zitiert
und die Bibliographie erstellt.
Der Code zum Bibliographiestil, den ich benutze,
ist ursprünglich von Michael Domhardt.

\subsubsection{Erstellung in der Hauptdatei}
\begin{verbatim}
\addsec{Quellen}
%\nocite{*}						% prints uncited works
\printbibliography[heading=none]
\end{verbatim}

\begin{center}
\begin{tabu} to \textwidth{>{\ttfamily}X[l] X[4l]}
	\toprule
	addsec\{...\}	& Kapitelbenennung ohne Nummer, wird im Inhaltsverzeichnis angezeigt\\
	nocite\{*\}		& falls unzitierte Werke auch aufgelistet werden sollen\\
	\bottomrule
\end{tabu}
\end{center}