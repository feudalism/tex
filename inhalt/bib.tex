\subsubsection{Einstellungen in der Präambel}
Um die Literaturverzeichnis zu erstellen, muss man in der Präambel
das Paket \texttt{biblatex} laden.
Dazu sind die folgenden Optionen empfohlen.
Ansonsten kann man in die \texttt{biblatex} Dokumentation
anschauen.

\begin{verbatim}
\usepackage[backend=biber,
    sorting=nty,
    date=year,
    sortcites=true,
    ]{biblatex}
\end{verbatim}

\begin{center}
\begin{tabu} to \textwidth{>{\ttfamily}X[l] X[4l]}
	\toprule
	sorting=nty	& sorts entries by name, title, year\\
	date=year	& nur das Veröffentlichungsjahr wird gezeigt, und
		nicht das vollständige Datum\\
	sortcites=true	& obige \texttt{nty} Option wird umgesetzt\\
	\bottomrule
\end{tabu}
\end{center}

Die Informationen (Angaben zu Autor, Jahr, Titel, Auflage usw.)
werden in einer \texttt{.bib} Datei aufbewahrt und müssten
ebenfalls in der Präambel geladen werden.

\begin{verbatim}
\addbibresource{./preamble/ref.bib}
\end{verbatim}
	
Man kann an dieser Stelle auch den Stil der Bibliographie einstellen.
Z.B. nach APA, MLA, usw.
\begin{verbatim}
\input{./preamble/bibstyle}
\end{verbatim}


Ich habe bisher immer nach DIN 1505 zitiert
und die Bibliographie erstellt.
Der Code zum Bibliographiestil, den ich benutze,
ist ursprünglich von Michael Domhardt.

\subsubsection{Erstellung in der Hauptdatei}
\begin{verbatim}
\addsec{Quellen}
%\nocite{*}						% prints uncited works
\printbibliography[heading=none]
\end{verbatim}

\begin{center}
\begin{tabu} to \textwidth{>{\ttfamily}X[l] X[4l]}
	\toprule
	addsec\{...\}	& Kapitelbenennung ohne Nummer, wird im Inhaltsverzeichnis angezeigt\\
	nocite\{*\}		& falls unzitierte Werke auch aufgelistet werden sollen\\
	\bottomrule
\end{tabu}
\end{center}