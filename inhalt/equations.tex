Mathematische Formeln oder Gleichungen können auch 
mitten im Absatz auftachen, z.B. $\sum_{n=1}^\infty a^i = \infty$,
wenn die Gleichung von den Zeichen \verb|$ ... $|
umgeben ist.

\begin{verbatim}
auftauchen, z.B. $\sum_{n=1}^\infty a^i = \infty$, wenn die ...
\end{verbatim}~ 

Auf einer neuen Zeile können die Gleichungen
von einer mathematischen Umgebungen umgeben werden.\\

Formel mit Nummerierung \verb|\begin{align} ... \end{align}| (Gleichung \ref{eq:formula})
\begin{align}
\int_{-\infty}^{+\infty} e^{-x^2} dx = \left( 6 \sum_{n=1}^{\infty} \frac{1}{n^2} \right)^\frac{1}{4}
\label{eq:formula}
\end{align}~

Formel ohne Nummerierung \verb|\begin{align*} ... \end{align*}|
\begin{align*}
\int_{-\infty}^{+\infty} e^{-x^2} dx = \left( 6 \sum_{n=1}^{\infty} \frac{1}{n^2} \right)^\frac{1}{4}
\end{align*}~

Formeln sind auch mithilfe der default Umgebung \verb|equation| realisierbar.
Allerdings bevorzuge ich \texttt{align}, besonders bei
mehrzeiligen Gleichungen. In \texttt{align} lassen sich nämlich
die Zeilen an dem \texttt{=} Zeichen besser ausrichten.
\begin{verbatim}
\begin{align*}
    1 + 1  &= 2\\
    341151 + 3262424 &= 3
\end{align*}
\end{verbatim}
\begin{align*}
1 + 1 &= 2\\
341151 + 3262424 &= 3
\end{align*}