\subsection{Beispieltext mit Fußnote}
In diesem Abschnitt wird die nichtparametrische Systemidentifikation
eingeführt.
Es folgt die Motivation zur Berechnung im Frequenzbereich,
insbesondere mittels der Fast-Fourier-Trans\-formation.
Eine Annäherung von den Sinus-Signalen mithilfe
der Curve Fitting Methode wird ebenfalls vorgestellt%
\footnote{Dies ist eine Fußnote!}.

\begin{verbatim}
... ebenfalls vorgestellt%
\footnote{Dies ist eine Fußnote!}.
\end{verbatim}

Schreibt man \verb|\footnote{...}| 
auf einer neuen Zeile im Editor, wird im Text nach dem
letzten Wort ein Leerzeichen erstellt.
Um dies zu vermeiden, kann man am Ende der vorherigen Zeile
\% hinschreiben.
Dies unterdruckt jegliche ``whitespace character''
bis zum Anfang der nächsten Zeile.\\

\subsection{Listen}
\begin{itemize}
\item Item1
\item Item2
\end{itemize}~

\begin{enumerate}
\item Item1
\item Item2
\end{enumerate}~

\begin{verbatim}
\begin{itemize}
\item Item1
\item Item2
\end{itemize}

\begin{enumerate}
\item Item1
\item Item2
\end{enumerate}
\end{verbatim}

Es gibt ein Paar Vorformatierungen die man in der Präambel durchführen kann, z.B. mithilfe des Pakets \texttt{enumitem}.
\begin{verbatim}
\usepackage{enumitem}
\setlist{nolistsep}    % no spacing between list items
\setlist[enumerate,1]
		% list indentation
		{labelindent=0.5cm,leftmargin=*}
\end{verbatim}

\subsection{Zitationen, Verweise}
Hier mal eine Beispielzitation \cite{online:uppsala}.
\begin{verbatim}
Hier mal eine Beispielzitation \cite{online:uppsala}.\\
\end{verbatim}~

Und ein Verweis auf Kapitel \ref{ch:organisation}.
	\begin{verbatim}
	\section{Organisation}
	\label{ch:organisation}
	    ...
	
	\section{Beispiele}
	    ....
	    Und ein Verweis auf Kapitel \ref{ch:organisation}.
	\end{verbatim}~

\subsection{Formeln}
Mathematische Formeln oder Gleichungen können auch 
mitten im Absatz auftachen, z.B. $\sum_{n=1}^\infty a^i = \infty$,
wenn die Gleichung von den Zeichen \verb|$ ... $|
umgeben ist.

\begin{verbatim}
auftauchen, z.B. $\sum_{n=1}^\infty a^i = \infty$, wenn die ...
\end{verbatim}~ 

Auf einer neuen Zeile können die Gleichungen
von einer mathematischen Umgebungen umgeben werden.\\

Formel mit Nummerierung \verb|\begin{align} ... \end{align}| (Gleichung \ref{eq:formula})
\begin{align}
\int_{-\infty}^{+\infty} e^{-x^2} dx = \left( 6 \sum_{n=1}^{\infty} \frac{1}{n^2} \right)^\frac{1}{4}
\label{eq:formula}
\end{align}~

Formel ohne Nummerierung \verb|\begin{align*} ... \end{align*}|
\begin{align*}
\int_{-\infty}^{+\infty} e^{-x^2} dx = \left( 6 \sum_{n=1}^{\infty} \frac{1}{n^2} \right)^\frac{1}{4}
\end{align*}~

Formeln sind auch mithilfe der default Umgebung \verb|equation| realisierbar.
Allerdings bevorzuge ich \texttt{align}, besonders bei
mehrzeiligen Gleichungen. In \texttt{align} lassen sich nämlich
die Zeilen an dem \texttt{=} Zeichen besser ausrichten.
\begin{verbatim}
\begin{align*}
    1 + 1  &= 2\\
    341151 + 3262424 &= 3
\end{align*}
\end{verbatim}
\begin{align*}
1 + 1 &= 2\\
341151 + 3262424 &= 3
\end{align*}

\subsection{Abbildungen}
Hier sieht man ein Beispielbild
(Abbildung \ref{fig:bsp1}).

\begin{verbatim}
... Beispielbild (Abbildung \ref{fig:bsp1}).

\begin{figure}[H]
    \centering
    \includegraphics[width=0.6\textwidth]{unilogo}
    \caption{Beispielabbildung}
    \label{fig:bsp1}
\end{figure}
\end{verbatim}

% [H] tells LaTeX to put the figure here,
% if not, LaTeX may push the figure to a 'more optimal' location
\begin{figure}[H]
\centering
\includegraphics[width=0.6\textwidth]{unilogo}
\caption{Beispielabbildung}
\label{fig:bsp1}
\end{figure}

\subsection{Tabellen}
Basic Tabellenaufbau mit dem \texttt{tabu}-Paket.
Eine Beispieltabelle ist in Tabelle \ref{tab:fit-table} zu finden.
\texttt{c} bedeutet, dass der Text in der Spalte zentriert ist.
Alternativ kann man stattdessen \texttt{l} oder \texttt{r} schreiben
(für Links- bzw. Rechtsausrichtung).

\begin{verbatim}
... ist in Tabelle \ref{tab:labelname} zu finden. ...\\

\begin{table}[H]
    \caption{Tabelle basic}
    \label{tab:labelname}
    \centering
    \begin{tabu} to \textwidth{cccc}
        \toprule
        Col1 & Col2 & Col3 & Col4\\
        \midrule
        1 & 2 & 3 & 4\\
        5 & 6 & 7 & 8\\
        \bottomrule
    \end{tabu}
\end{table}
\end{verbatim}

\begin{table}[H]
	\caption{Tabelle basic}
	\label{tab:fit-table}
	\centering
	\begin{tabu} to \textwidth{cccc}
		\toprule
		Col1 & Col2 & Col3 & Col4\\
		\midrule
		1 & 2 & 3 & 4\\
		5 & 6 & 7 & 8\\
		\bottomrule
	\end{tabu}
\end{table}~



Die obigen Tabelle \ref{tab:fit-table} hat Spaltenbreiten,
die dem Inhalt angepasst sind.
Um mehr Kontrolle über die Spaltenbreite zu haben, kann man 
die sogenannten X-columns verwenden.
Die folgende Tabelle \ref{tab:fullw-table} hat 
gleichmäßig-verteilte Spalten über die gesamte Textfeldbreite.

\begin{verbatim}
\begin{tabu} to \textwidth{X[c] X[c] X[c] X[c]}
    ....
\end{table}
\end{verbatim}

\begin{table}[H]
	\caption{Tabelle über voller Textlänge}
	\label{tab:fullw-table}
	\centering
	\begin{tabu} to \textwidth{X[c] X[c] X[c] X[c]}
		\toprule
		Col1 & Col2 & Col3 & Col4\\
		\midrule
		1 & 2 & 3 & 4\\
		5 & 6 & 7 & 8\\
		\bottomrule
	\end{tabu}
\end{table}~



Die Angabe \verb|\textwidth| kann auch z.B. zu \verb|0.6\textwidth|
umgeschrieben werden, um die Tabellenbreite zu beschränken.

\begin{verbatim}
\begin{tabu} to 0.6\textwidth{X[r] X[r] X[r] X[r]}
     ...
\end{verbatim}

\begin{table}[H]
	\caption{Tabelle über 60\% Textlänge}
	\label{tab:partial-table}
	\centering
	\begin{tabu} to 0.6\textwidth{X[r] X[r] X[r] X[r]}
		\toprule
		Col1 & Col2 & Col3 & Col4\\
		\midrule
		1 & 2 & 3 & 4\\
		5 & 6 & 7 & 8\\
		\bottomrule
	\end{tabu}
\end{table}~



Bisher hatten die Spalten alle die gleiche Breite.
Um z.B. die Breiten im Verhältnis 3:3:1:1 darzustellen, verwende
\begin{verbatim}
\begin{tabu} to 0.6\textwidth{X[3c] X[3c] X[1c] X[1c]}
    ...
\end{verbatim}

\begin{table}[H]
	\caption{Tabelle mit variablen Spaltenbreite}
	\label{tab:var-colw-table}
	\centering
	\begin{tabu} to 0.6\textwidth{X[3c] X[3c] X[1c] X[1c]}
		\toprule
		Col1 & Col2 & Col3 & Col4\\
		\midrule
		1 & 2 & 3 & 4\\
		5 & 6 & 7 & 8\\
		\bottomrule
	\end{tabu}
\end{table}~



\begin{verbatim}
\begin{tabu} to \textwidth{+l^r^r^r}
    \rowstyle{\bfseries}
    % Inhalt der ersten Zeile
    Col1 & Col2 & Col3 & Col4\\
    ...
\end{verbatim}

\begin{table}[H]
	\caption{Tabelle mit fetten Überschriften (Zeile)}
	\label{tab:bold-table-row}
	\centering
	\begin{tabu} to \textwidth{+l^r^r^r}
		\toprule
		\rowstyle{\bfseries}
		Col1 & Col2 & Col3 & Col4\\
		\midrule
		1 & 2 & 3 & 4\\
		5 & 6 & 7 & 8\\
		\bottomrule
	\end{tabu}
\end{table}~



\begin{verbatim}
\begin{tabu} to \textwidth{*l|rrr}
    ...
\end{verbatim}


\begin{table}[H]
	\caption{Tabelle mit fetten Überschriften (Spalte)}
	\label{tab:bold-table-col}
	\centering
	\begin{tabu} to \textwidth{*l|rrr}
		H1 & 11 & 21 & 31\\
		H2 & 2 & 3 & 4\\
		H3 & 6 & 7 & 8\\
	\end{tabu}
\end{table}
